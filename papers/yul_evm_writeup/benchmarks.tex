\subsection{Without Pointers}

\todo{overhead here but it's acceptable. optimizer does a lot. these
  contracts don't touch memory, so their tracers are all the trivial
  ones. this gives a sense of the baseline overhead we add.}

\begin{figure}[hbtp]
    \caption{Benchmarks with Trivial Tracers}
    \label{data.1}
    \resizebox{\columnwidth}{!}{
      \csvautotabular{small_bench.csv}
    }
\end{figure}


\subsection{With Pointers}

\todo{lots of overhead; it's cheaper to turn off gc, because you're doing
  more work collecting than you get repaid for}

\begin{figure}[hbtp]
    \caption{Benchmarks with Pointer Fields}
    \label{data.2}
    \resizebox{\columnwidth}{!}{
      \csvautotabular{medium_bench.csv}
    }
\end{figure}


\subsection{Linked List}

\todo{the theory is that once you have enough in storage, the rebates take
  over and are more than the cost of collecting them. these tests are with
  a simple linked list of 4 and 8 items. if you imagine a doubly linked
  list backing a priority / de / queue, it'll be faster. that datastructure
  is known to be hard to implement in the sort of associative mappings that
  are built into solidity, and its use might arise very naturally from a
  smart contract that processes queries in arrival order.}

\begin{figure}[hbtp]
    \caption{Benchmarks With Linked Lists}
    \label{data.3}
    \resizebox{\columnwidth}{!}{
      \csvautotabular{ll_bench.csv}
    }
\end{figure}


\todo{Note that differences in gas are roughly comparable to the ratio of
  filesize differences.}

\begin{figure}[hbtp]
    \caption{Sizes of Optimized GC and non-GC Linked List Benchmarks}
    \label{data.4}
    \resizebox{\columnwidth}{!}{
      \csvautotabular{sizes.csv}
    }
\end{figure}
