\subsection{Optimizations}

As described above, the execution pipeline invokes the Yul optimizer built
into \solc. That is required, at least, because the compiler does not
attempt to obey the Yul restriction on the number of local variables in the
process of creating temporary variables. For all the example programs we
have wanted to run to date, the optimizer is sufficient to fix this
issue. It is quite aggressive, and its presence in the pipeline goes a long
way to keeping gas costs low.

That said, information available at compile time from the AST, typing, and
structural information of an Obsidian program, as well as knowledge about
the intended layout in memory and its semantics, allows for richer
structural optimizations than the Yul optimizer can perform and that we
have not implemented in our prototype so far. We believe that each of these
will contribute to a more cost-effective runtime, but fine tuning an
implementation of any one of them to achieve its best result is the subject
of future work.

In the currently emitted code, the translation of each transaction checks
the location of its instance before each and every interaction with either
memory or storage. The answer to this check is always the same within a
given execution of the transaction, so all but the first check is
redundant. A solution to this is to emit two Yul transactions per Obsidian
transaction: one that operates entirely on memory and one on storage. This
means that before calling a transaction you need to check once where its
instance lives, but after that there are no additional checks.

\todo{describe other big picture optimizations, beyond the scope of the Yul
  optimizer, here}

\subsection{Features}

Garbage collection in the current prototype is manual, where we extend the
Obsidian language to include \obssrc{retain} and \obssrc{release} methods
for every contract in the style of C\# \cite{todo}. This is sufficient for
a proof of concept, but it is more of a burden on the programmer than we
envision: it is well established that explicit memory controls both become
a substantial cognitive effort even in relatively small programs as well as
a source of bugs. \cite{todo} In a future version of the Obsidian
prototype, we plan to implement automatic garbage collection in the style
of \todo{}. \cite{todo}

We plan to extend the translation of Obsidian into Yul to include a richer
space of language features, including but not limited to:
\begin{enumerate} %% from milestones at the very beginning of the project
\item ethereum addresses
\item basic data structures like arrays and mappings
\item the \obssrc{revert} construct
\item dynamic state tests
\item dynamic contract allocation
\item support for ether as an asset type
\item sending and receiving ether in invocations
\end{enumerate}
These features have been on our roadmap since the beginning of this
project, but the implementation of them was eclipsed by the layout
described in \ref{sec:layout}, which were both our main concern and more
intricate than anticipated.
